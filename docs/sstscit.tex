\documentclass{article}

\usepackage{custom}

\usepackage[utf8]{inputenc} % allow utf-8 input
\usepackage[T1]{fontenc}    % use 8-bit T1 fonts
\usepackage{hyperref}       % hyperlinks
\usepackage{url}            % simple URL typesetting
\usepackage{booktabs}       % professional-quality tables
\usepackage{amsfonts}       % blackboard math symbols
\usepackage{nicefrac}       % compact symbols for 1/2, etc.
\usepackage{microtype}      % microtypography
\usepackage{lipsum}

\usepackage{graphicx}
\usepackage{listings}
\usepackage{color}

\definecolor{dkgreen}{rgb}{0,0.6,0}
\definecolor{gray}{rgb}{0.5,0.5,0.5}
\definecolor{mauve}{rgb}{0.58,0,0.82}

\lstset{frame=tb,
  aboveskip=3mm,
  belowskip=3mm,
  showstringspaces=false,
  columns=flexible,
  basicstyle={\small\ttfamily},
  numbers=none,
  numberstyle=\tiny\color{gray},
  keywordstyle=\color{blue},
  commentstyle=\color{dkgreen},
  stringstyle=\color{mauve},
  breaklines=true,
  breakatwhitespace=true,
  tabsize=4,
  escapeinside={<@}{@>}
}

\title{SST-SystemC Interoperability Toolkit}

\author{
  Sabbir Ahmed
}

\begin{document}
  \maketitle

  \begin{abstract}
    SST-SystemC Interoperability Toolkit (SSTSCIT) is a collection of header files developed to
    provide interoperability between Structural Simulation Toolkit (SST) and SystemC. SST is a
    parallel event based simulation framework that allows custom and vendor models to be
    interconnected to create a simulation environment \cite{sst}. SystemC is a system-level modeling
    language composed of C++ classes and macros \cite{sysc}. SSTSCIT aims to provide the capability
    to interoperate the two systems without interfering with any of the kernels by concealing the
    communication protocols in black box interfaces. This project provides a demonstration of the
    interoperability by simulating a traffic intersection, where the traffic lights are determined
    by SystemC processes.
  \end{abstract}

  \section{Introduction}
  This collection of header files provides methods to transmit and receive signals between SST
  components and SystemC modules. The toolkit provides a black box interface that can be interfaced
  with both SST and SystemC via their internal communication transports. 

  \begin{figure}[!h]
    \centering
    \includegraphics[width=6in]{diagrams/comm.png}
    \caption{Components of SSTSCIT}
    \label{fig:comm}
  \end{figure}

  \section{Components} \label{sec:comp}

    \subsection{Black Box Interface}
    The black box interface consists of:
    \begin{enumerate}
        \item A SystemC driver
        \item An SST component
    \end{enumerate}

    Each SystemC modules must have their corresponding driver file to interoperate within the black
    box interface. It is possible to interoperate multiple SystemC modules with a single driver
    file. However, the additional communication lines must be accounted for in the corresponding
    black box SST component.

    The toolkit includes a Python class that generates the boilerplate code required for the black
    box interface.

  \begin{figure}[!h]
    \centering
    \includegraphics[width=5.5in]{diagrams/data_flow.png}
    \caption{Black Box Interface Data Flow Diagram}
    \label{fig:data_flow}
  \end{figure}

  \section{Communication}

    \subsection{Inter-Black Box Communication}
    The two components inside the black box interface are spawned in the same node and therefore
    communicate via interprocess communication (IPC) transports. The following is a list of
    supported IPC transports:
    \begin{enumerate}
      \item Unix domain sockets
      \item ZeroMQ
    \end{enumerate}

    It is possible to add custom IPC protocols to the interface by implementing a derived class of
    \lstinline{sigutils::SignalIO} with customized sending and receiving methods.

    \subsection{SST-Black Box Communication}
    An SST model can interface the black box via standard SST links.

    The following snippets demonstrate an SST link transmitting a unidirectional signal from the SST
    environment to the black box interface.

\begin{lstlisting}[language=C++]
// parent_sst.cpp

// register a string event in the class declaration
SST_ELI_DOCUMENT_PORTS(
    { "demo_din", "Demo model data in", { "sst.Interfaces.StringEvent" }},
    ...
)

// initialize the link in the class declaration
SST::Link *demo_din;

// set up the SST link in the constructor
demo_din = configureLink("demo_din");

// trigger the event in the clocked function
demo_din->send(new SST::Interfaces::StringEvent(...));
\end{lstlisting}

\begin{lstlisting}[language=C++]
// blackboxes/demo.cpp

// register the same string event in the class declaration
SST_ELI_DOCUMENT_PORTS(
    { "demo_din", "Demo model data in", { "sst.Interfaces.StringEvent" }},
    ...
)

// initialize the same link in the class declaration
SST::Link *demo_din;

// set up the SST link in the constructor with an event handler
demo_din = configureLink(
    "demo_din",
    new SST::Event::Handler<demo>(this, &demo::handle_event)
);

// receive and parse the event in the event handler
void demo::handle_event(SST::Event *ev) {
    auto *se = dynamic_cast<SST::Interfaces::StringEvent *>(ev);
    if (se) {
        std::string _data_in = se->getString();
        ...
    }
    delete ev;
}
\end{lstlisting}

    \subsection{SystemC-Black Box Communication}
    A SystemC module can be interfaced by a standard source file inclusion.

  \section{Proof of Concept: Traffic Intersection Simulation}
  A simulation model has been developed to demonstrate the project. The model simulates a traffic
  intersection controlled by two traffic lights. A flow of traffic is simulated through a road only
  when its traffic light generates a green or yellow light with the other generating a red light.
  The number of cars in a traffic flow is represented by random number generators.

  \begin{figure}[!h]
    \centering
    \includegraphics[width=3.5in]{diagrams/intersection.png}
    \caption{Simple Two-Road Intersection}
    \label{fig:intersection}
  \end{figure}

  The concept of this simulation is derived from the original project that established
  interoperability between SST and PyRTL - a Python based hardware description
  language.\cite{pyrtl-sst}

    \subsection{SystemC Drivers}
    The simulation project includes a SystemC module and its driver, \lstinline{traffic_light_fsm},
    that interacts with the SST component \lstinline{traffic_light}. The module is a clock-driven
    FSM of three states representing the three colors of a traffic light: green, yellow and red. The
    FSM proceeds to the next state when indicated by its internal counter initialized in the
    beginning. The input variables to the module include: the three durations for the three colors
    of the light, \lstinline{green_time}, \lstinline{yellow_time} and \lstinline{red_time}, the
    preset variable \lstinline{load} to initialize the FSM, and \lstinline{start_green} to indicate
    if the first state should be ``green'' or ``red''.

    \begin{figure}[!h]
      \centering
      \includegraphics[width=4in]{diagrams/fsm.png}
      \caption{Traffic Light Finite State Machine}
      \label{fig:fsm}
    \end{figure}

    \subsection{SST Components}
    The project also consists of three SST components: \lstinline{car_generator},
    \lstinline{traffic_light_controller} and \lstinline{intersection}. All the components with the
    exception of most of \lstinline{traffic_light_controller} were inherited from the original
    project.

      \subsubsection{car\_generator}
      The \lstinline{car_generator} component consists of a random number generator that yields $0$
      or $1$. The output is redirected to \lstinline{intersection} via SST links.

      \subsubsection{traffic\_light}

      The \lstinline{traffic_light} component generates the light colors of the traffic lights using
      a simple finite state machine (FSM). The component delegates the FSM portion of its algorithm
      to the SystemC module \lstinline{traffic_light_fsm} and inter-procedurally communicates with
      it via UNIX domain sockets. The component initializes the FSM with the SST parameters and
      sends its output to \lstinline{intersection} via SST links every clock cycle.

      \subsubsection{intersection}
      \lstinline{intersection} is the main driver of the simulation. The component is able to handle
      $n$ instances of \lstinline{traffic_light} subcomponents and therefore expects the same number
      of \lstinline{car_generator} instances. For the purposes of this simulation, two instances of
      the subcomponent pairs are set up. The driver keeps track of the number of cars generated and
      the color of the light per clock cycle for each subcomponent pairs and stores them in local
      variables. The variables are summarized in the end to generate statistics about the
      simulation.

      The component does not check for any collisions in the intersection, i.e. comparing if both
      the \lstinline{traffic_light} subcomponents yield ``green'' during the same clock cycle. The
      SST Python module is responsible for setting up the \lstinline{traffic_light} components with
      the proper initial values.

      \begin{figure}[!h]
        \centering
        \includegraphics[width=4in]{diagrams/intersection_comp.png}
        \caption{Simple Two-Road Intersection}
        \label{fig:intersection_comp}
      \end{figure}

      \newpage
      \clearpage
    \subsection{Example Simulation}

\begin{lstlisting}
<@\textcolor{dkgreen}{traffic\_light-Traffic Light 0 ->}@> GREENTIME=30, YELLOWTIME=3, REDTIME=63, STARTGREEN=0
<@\textcolor{dkgreen}{traffic\_light-Traffic Light 1 ->}@> GREENTIME=60, YELLOWTIME=3, REDTIME=33, STARTGREEN=1
<@\textcolor{mauve}{car\_generator-Car Generator 0 ->}@> Minimum Delay Between Cars=3s, Random Number Seed=151515
<@\textcolor{mauve}{car\_generator-Car Generator 1 ->}@> Minimum Delay Between Cars=5s, Random Number Seed=239478
<@\textcolor{blue}{intersection-Intersection ->}@> sim_duration=24 Hours
<@\textcolor{blue}{intersection-Intersection ->}@> Component is being set up.
<@\textcolor{dkgreen}{traffic\_light-Traffic Light 0 ->}@> Component is being set up.
<@\textcolor{dkgreen}{traffic\_light-Traffic Light 0 ->}@> Forking process "/home/sabbir/projects/sstscit/examples/intersection/build/traffic_light_fsm.o"...
<@\textcolor{dkgreen}{traffic\_light-Traffic Light 0 ->}@> Process "/home/sabbir/projects/sstscit/examples/intersection/build/traffic_light_fsm.o" successfully synchronized
<@\textcolor{dkgreen}{traffic\_light-Traffic Light 1 ->}@> Component is being set up.
<@\textcolor{dkgreen}{traffic\_light-Traffic Light 1 ->}@> Forking process "/home/sabbir/projects/sstscit/examples/intersection/build/traffic_light_fsm.o"...
<@\textcolor{dkgreen}{traffic\_light-Traffic Light 1 ->}@> Process "/home/sabbir/projects/sstscit/examples/intersection/build/traffic_light_fsm.o" successfully synchronized
<@\textcolor{blue}{intersection-Intersection ->}@> --------------------------------------
<@\textcolor{blue}{intersection-Intersection ->}@> -------- SIMULATION INITIATED --------
<@\textcolor{blue}{intersection-Intersection ->}@> --------------------------------------
<@\textcolor{blue}{intersection-Intersection ->}@> Hour | Total Cars TL0 | Total Cars TL1
<@\textcolor{blue}{intersection-Intersection ->}@> -----+----------------+---------------
<@\textcolor{blue}{intersection-Intersection ->}@>    1 |              618 |            369
<@\textcolor{blue}{intersection-Intersection ->}@>    2 |             1238 |            719
<@\textcolor{blue}{intersection-Intersection ->}@>    3 |             1851 |           1074
<@\textcolor{blue}{intersection-Intersection ->}@>    4 |             2432 |           1426
<@\textcolor{blue}{intersection-Intersection ->}@>    5 |             3041 |           1774
<@\textcolor{blue}{intersection-Intersection ->}@>    6 |             3688 |           2121
<@\textcolor{blue}{intersection-Intersection ->}@>    7 |             4290 |           2467
<@\textcolor{blue}{intersection-Intersection ->}@>    8 |             4892 |           2813
<@\textcolor{blue}{intersection-Intersection ->}@>    9 |             5467 |           3175
<@\textcolor{blue}{intersection-Intersection ->}@>   10 |             6054 |           3525
<@\textcolor{blue}{intersection-Intersection ->}@>   11 |             6644 |           3885
<@\textcolor{blue}{intersection-Intersection ->}@>   12 |             7228 |           4233
<@\textcolor{blue}{intersection-Intersection ->}@>   13 |             7813 |           4607
<@\textcolor{blue}{intersection-Intersection ->}@>   14 |             8435 |           4973
<@\textcolor{blue}{intersection-Intersection ->}@>   15 |             9047 |           5337
<@\textcolor{blue}{intersection-Intersection ->}@>   16 |             9656 |           5691
<@\textcolor{blue}{intersection-Intersection ->}@>   17 |            10255 |           6059
<@\textcolor{blue}{intersection-Intersection ->}@>   18 |            10843 |           6428
<@\textcolor{blue}{intersection-Intersection ->}@>   19 |            11448 |           6791
<@\textcolor{blue}{intersection-Intersection ->}@>   20 |            12025 |           7140
<@\textcolor{blue}{intersection-Intersection ->}@>   21 |            12617 |           7499
<@\textcolor{blue}{intersection-Intersection ->}@>   22 |            13225 |           7867
<@\textcolor{blue}{intersection-Intersection ->}@>   23 |            13807 |           8223
<@\textcolor{blue}{intersection-Intersection ->}@>   24 |            14400 |           8580
<@\textcolor{blue}{intersection-Intersection ->}@> 
<@\textcolor{blue}{intersection-Intersection ->}@> -------------------------------------------
<@\textcolor{blue}{intersection-Intersection ->}@> ---------- SIMULATION STATISTICS ----------
<@\textcolor{blue}{intersection-Intersection ->}@> -------------------------------------------
<@\textcolor{blue}{intersection-Intersection ->}@> Traffic Light | Total Cars | Largest Backup
<@\textcolor{blue}{intersection-Intersection ->}@> --------------+------------+---------------
<@\textcolor{blue}{intersection-Intersection ->}@>              0 |       14400 |            18
<@\textcolor{blue}{intersection-Intersection ->}@>              1 |        8581 |             7
<@\textcolor{blue}{intersection-Intersection ->}@> Destroying Intersection...
<@\textcolor{mauve}{car\_generator-Car Generator 1 ->}@> Destroying Car Generator 1...
<@\textcolor{mauve}{car\_generator-Car Generator 0 ->}@> Destroying Car Generator 0...
<@\textcolor{dkgreen}{traffic\_light-Traffic Light 0 ->}@> Destroying Traffic Light 0...
<@\textcolor{dkgreen}{traffic\_light-Traffic Light 1 ->}@> Destroying Traffic Light 1...
Simulation is complete, simulated time: 86.4 Ks

\end{lstlisting}
\newpage

  \begin{thebibliography}{5}

    \bibitem{sst} SST Simulator - The Structural Simulation Toolkit. \texttt{sst-simulator.org}.

    \bibitem{sysc} 1666-2011 - IEEE Standard for Standard SystemC Language Reference Manual. IEEE, 9
    Jan. 2012, \texttt{standards.ieee.org/standard/1666-2011.html}.

    \bibitem{pyrtl-sst} Mrosky, Robert P, et al.
    \textit{Creating Heterogeneous Simulations with SST and PyRTL}.


  \end{thebibliography}

\end{document}
