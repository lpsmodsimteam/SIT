\documentclass{article}

\usepackage{custom}

\usepackage[utf8]{inputenc} % allow utf-8 input
\usepackage[T1]{fontenc}    % use 8-bit T1 fonts
\usepackage{hyperref}       % hyperlinks
\usepackage{url}            % simple URL typesetting
\usepackage{booktabs}       % professional-quality tables
\usepackage{amsfonts}       % blackboard math symbols
\usepackage{nicefrac}       % compact symbols for 1/2, etc.

\usepackage{graphicx}
\usepackage{listings}
\usepackage{color}
\usepackage[labelfont=bf]{caption}

\definecolor{dkgreen}{rgb}{0,0.6,0}
\definecolor{gray}{rgb}{0.5,0.5,0.5}
\definecolor{mauve}{rgb}{0.58,0,0.82}

\lstset{frame=tb,
  aboveskip=3mm,
  belowskip=3mm,
  showstringspaces=false,
  columns=flexible,
  basicstyle={\small\ttfamily},
  numbers=none,
  numberstyle=\tiny\color{gray},
  keywordstyle=\color{blue},
  commentstyle=\color{dkgreen},
  stringstyle=\color{mauve},
  breaklines=true,
  breakatwhitespace=true,
  tabsize=4,
  escapeinside={<@}{@>}
}

\title{Creating Heterogeneous Simulations with SST and SystemC}

\author{
  Sabbir Ahmed \\
  Booz Allen Hamilton \\
  ahmed\_sabbir@bah.com
  \And
  Noel S. Wheeler \\
  Laboratory for Physical Sciences \\
  nwheeler@lps.umd.edu
  \And
  Robert P. Mrosky \\
  Laboratory for Physical Sciences \\
  rmrosk1@lps.umd.edu
}

\begin{document}
  \maketitle

  \begin{abstract}
    Implementing new computer system designs involves careful study of both programming models and
    hardware design and organization, a process that frequently introduces distinct challenges.
    Hardware and software definitions are often simulated to undertake these difficulties.
    Structural Simulation Toolkit (SST), a parallel event-based simulation framework that allows
    custom and vendor models to be interconnected to create a system simulation \cite{sst}, is one
    such toolkit. However, SST must be able to support models implemented in various frameworks and
    languages. SystemC is a popular hardware-level modeling language composed of C++ classes and
    macros \cite{sysc}. Establishing communication with SystemC modules would allow SST to interface
    numerous existing synthesizable hardware models. SST-SystemC Interoperability Toolkit (SSTI)
    is a library developed to provide interoperability between SST and SystemC. SSTI aims to
    achieve this capability in a modular design without interfering with the kernels by concealing
    the communication protocols in black box interfaces. This project includes a demonstration of
    the interoperability by simulating a vehicular traffic intersection with traffic lights driven
    by SystemC processes. The modular implementation of the black box interface allowed for
    sufficient flexibility in establishing communication between the different systems. This design
    can, therefore, be configured to interoperate SST with various model simulation frameworks and
    even hardware to achieve further heterogeneity.
  \end{abstract}

  \section{Introduction}  
  The increasing size and complexity of systems require engineers heavily rely on simulation
  techniques during the development phases. Typically, simulations of these complex systems require
  both custom and off-the-shelf logic functionality in ASICs or FPGAs. High-level commercial tools
  simulate and model these components in their native environments. On the other side, developers
  create the register transfer level (RTL) models representing the systems to simulate them with
  computer-aided design (CAD) tools and test benches. These duplicative strategies require a method
  that simulates the entire system in one heterogeneous model.

  Successful attempts have been made to establish interoperability between Structural Simulation
  Toolkit (SST) and the Python-based RTL language, PyRTL \cite{pyrtl-sst}. This project establishes
  interoperability between SST and SystemC and demonstrates its extensibility to further systems due
  to the modular design.

  SST is an event-based framework that has the capabilities to simulate not only functionality but
  timing, power or any other information required. Each SST components can be assigned a clock to
  synchronize tasks. They communicate events with each other via SST links by triggering their
  corresponding event handlers. The SST models are constructed in C++ and consist of the
  functionality of the element, the definition of each links' ports and the event handlers. The
  models are connected and initialized through the SST Python module. SystemC is a set of C++
  classes and macros that deliberately mimics hardware description languages like VHDL and Verilog.
  The system-level modeling language provides an event-driven simulation kernel along with signals,
  events and synchronization primitives.

  Implementing a heterogeneous system to synchronize signals and events between the two kernels
  would allow the developers to work cooperatively and efficiently. This paper provides a
  demonstration of the interoperability by simulating a vehicular traffic intersection with traffic
  lights driven by SystemC processes.

  \section{Black Box Interface}
  SSTI conceals the communication implementation in black box driver files. This strategy allows
  the SST component to connect with the SystemC process via SST links as if it were a component
  itself.

  \begin{figure}[!h]
    \centering
    \includegraphics[width=6.5in]{diagrams/comm.png}
    \caption{Components of SSTI}
    \label{fig:comm}
  \end{figure}

  The interface consists of:
  \begin{enumerate}
    \item a SystemC driver
    \item an SST component
    \item configurations for inter-black box communication
  \end{enumerate}

  Each SystemC modules must have their corresponding driver file to interoperate within the black
  box interface. It is possible to interoperate multiple SystemC modules with a single driver
  file. However, the additional communication lines must be accounted for in the corresponding
  black box SST component.

  The interface also includes a header file with configurations for the communication between the
  components. These configurations include the number of elements being shared and their
  corresponding data structure indices.

  The toolkit includes a Python class that generates the boilerplate code required for the black
  box interface.

  Both the SST component and the SystemC driver has to be compiled separately.

  \begin{figure}[!h]
    \centering
    \includegraphics[width=5in]{diagrams/data_flow.png}
    \caption{Black Box Interface Data Flow Diagram; Arrows Highlighted in Red Indicate
    Communication Signals}
    \label{fig:data_flow}
  \end{figure}

  \section{Communication}

    \subsection{Inter-Black Box Communication} \label{sec:ipc}
    The data is represented in a standard vector of strings with the indices generated by the black
    box interface and then serialized with MessagePack methods \cite{msgpack}. The components inside
    the black box interface are spawned in the same node and therefore communicate via interprocess
    communication (IPC) transports. The following is a list of supported IPC transports:
    \begin{enumerate}
      \item Unix domain sockets
      \item ZeroMQ
    \end{enumerate}

    It is possible to integrate additional IPC protocols to the interface such as named pipes and
    shared memories.

    \subsection{SST-Black Box Communication}
    An SST component can interface the black box via standard SST links. The data is received as a
    \lstinline{SST::Interfaces::StringEvent} object which is casted to a standard string. SSTI
    provides a custom event handler as part of its black box interface to allocate the substring
    positions and lengths for the ports.

    \subsection{SystemC-Black Box Communication}

    A SystemC module can be interfaced by a standard source file inclusion. The driver does require
    the overridden entry point, \lstinline{sc_main}, to instantiate the module and set up the
    communication configurations.

  \section{Proof of Concept: Traffic Intersection Simulation}
  A simulation model has been developed to demonstrate the project. The model simulates a traffic
  intersection controlled by two traffic lights. A flow of traffic is simulated through a road only
  when its traffic light generates a green or yellow light with the other generating a red light.
  The number of cars in a traffic flow is represented by random number generators.

  \begin{figure}[!h]
    \centering
    \includegraphics[width=3.5in]{diagrams/intersection.png}
    \caption{Simple Two-Road Intersection}
    \label{fig:intersection}
  \end{figure}

  The concept of this simulation is derived from the original project that established
  interoperability between SST and PyRTL. \cite{pyrtl-sst}

    \subsection{SystemC Drivers}
    The simulation project includes a SystemC module and its driver, \lstinline{traffic_light_fsm},
    that interacts with the SST component \lstinline{traffic_light}. The module is a clock-driven
    FSM of three states representing the three colors of a traffic light: green, yellow and red. The
    FSM proceeds to the next state when indicated by its internal counter initialized in the
    beginning. The input variables to the module include: the three durations for the three colors
    of the light, \lstinline{green_time}, \lstinline{yellow_time} and \lstinline{red_time}, the
    preset variable \lstinline{load} to initialize the FSM, and \lstinline{start_green} to indicate
    if the first state should be ``green'' or ``red''.

    \begin{figure}[!h]
      \centering
      \includegraphics[width=4in]{diagrams/fsm.png}
      \caption{Traffic Light Finite State Machine}
      \label{fig:fsm}
    \end{figure}

    \subsection{SST Components}
    The project also consists of three SST components: \lstinline{car_generator},
    \lstinline{traffic_light_controller} and \lstinline{intersection}. All the components with the
    exception of most of \lstinline{traffic_light_controller} were inherited from the original
    project.

      \begin{figure}[!h]
        \centering
        \includegraphics[width=4.5in]{diagrams/intersection_comp.png}
        \caption{Simple Two-Road Intersection Represented by SST Components}
        \label{fig:intersection_comp}
      \end{figure}

      \subsubsection{car\_generator}
      The \lstinline{car_generator} component consists of a random number generator that yields $0$
      or $1$. The output is redirected to \lstinline{intersection} via SST links.

      \subsubsection{traffic\_light}
      The \lstinline{traffic_light} component generates the light colors of the traffic lights using
      a simple finite state machine (FSM). The component delegates the FSM portion of its algorithm
      to the SystemC module \lstinline{traffic_light_fsm} and inter-procedurally communicates with
      it via Unix domain sockets. The component initializes the FSM with the SST parameters and
      sends its output to \lstinline{intersection} via SST links every clock cycle.

      \subsubsection{intersection}
      \lstinline{intersection} is the main driver of the simulation. The component is able to handle
      $n$ instances of \lstinline{traffic_light} subcomponents and therefore expects the same number
      of \lstinline{car_generator} instances. For the purposes of this simulation, two instances of
      the subcomponent pairs are set up. The driver keeps track of the number of cars generated and
      the color of the light per clock cycle for each subcomponent pairs and stores them in local
      variables. The variables are summarized in the end to generate statistics about the
      simulation.

      The component does not check for any collisions in the intersection, i.e. comparing if both
      the \lstinline{traffic_light} subcomponents yield ``green'' during the same clock cycle. The
      SST Python module is responsible for setting up the \lstinline{traffic_light} components with
      the proper initial values.

    \subsection{Example Simulation}
    A sample output of the simulation has been generated and provided below. The SST components were
    linked along with the parameters in the SST Python module.

\begin{lstlisting}[caption={Sample Simulation Output}, captionpos=b]
<@\textcolor{dkgreen}{traffic\_light-Traffic Light 0 ->}@> GREENTIME=30, YELLOWTIME=3, REDTIME=63, STARTGREEN=0
<@\textcolor{dkgreen}{traffic\_light-Traffic Light 1 ->}@> GREENTIME=60, YELLOWTIME=3, REDTIME=33, STARTGREEN=1
<@\textcolor{mauve}{car\_generator-Car Generator 0 ->}@> Minimum Delay Between Cars=3s, Random Number Seed=151515
<@\textcolor{mauve}{car\_generator-Car Generator 1 ->}@> Minimum Delay Between Cars=5s, Random Number Seed=239478
<@\textcolor{blue}{intersection-Intersection ->}@> sim_duration=24 Hours
<@\textcolor{blue}{intersection-Intersection ->}@> Component is being set up.
<@\textcolor{dkgreen}{traffic\_light-Traffic Light 0 ->}@> Component is being set up.
<@\textcolor{dkgreen}{traffic\_light-Traffic Light 0 ->}@> Forking process "/path/to/traffic_light_fsm.o"...
<@\textcolor{dkgreen}{traffic\_light-Traffic Light 0 ->}@> Process "/path/to/traffic_light_fsm.o" successfully synchronized
<@\textcolor{dkgreen}{traffic\_light-Traffic Light 1 ->}@> Component is being set up.
<@\textcolor{dkgreen}{traffic\_light-Traffic Light 1 ->}@> Forking process "/path/to/traffic_light_fsm.o"...
<@\textcolor{dkgreen}{traffic\_light-Traffic Light 1 ->}@> Process "/path/to/traffic_light_fsm.o" successfully synchronized
<@\textcolor{blue}{intersection-Intersection ->}@> --------------------------------------
<@\textcolor{blue}{intersection-Intersection ->}@> -------- SIMULATION INITIATED --------
<@\textcolor{blue}{intersection-Intersection ->}@> --------------------------------------
<@\textcolor{blue}{intersection-Intersection ->}@> Hour | Total Cars TL0 | Total Cars TL1
<@\textcolor{blue}{intersection-Intersection ->}@> -----+----------------+---------------
<@\textcolor{blue}{intersection-Intersection ->}@>    1 |              618 |            369
<@\textcolor{blue}{intersection-Intersection ->}@>    2 |             1238 |            719
<@\textcolor{blue}{intersection-Intersection ->}@>    3 |             1851 |           1074
<@\textcolor{blue}{intersection-Intersection ->}@>    4 |             2432 |           1426
<@\textcolor{blue}{intersection-Intersection ->}@>    5 |             3041 |           1774
<@\textcolor{blue}{intersection-Intersection ->}@>    6 |             3688 |           2121
<@\textcolor{blue}{intersection-Intersection ->}@>    7 |             4290 |           2467
<@\textcolor{blue}{intersection-Intersection ->}@>    8 |             4892 |           2813
<@\textcolor{blue}{intersection-Intersection ->}@>    9 |             5467 |           3175
<@\textcolor{blue}{intersection-Intersection ->}@>   10 |             6054 |           3525
<@\textcolor{blue}{intersection-Intersection ->}@>   11 |             6644 |           3885
<@\textcolor{blue}{intersection-Intersection ->}@>   12 |             7228 |           4233
<@\textcolor{blue}{intersection-Intersection ->}@>   13 |             7813 |           4607
<@\textcolor{blue}{intersection-Intersection ->}@>   14 |             8435 |           4973
<@\textcolor{blue}{intersection-Intersection ->}@>   15 |             9047 |           5337
<@\textcolor{blue}{intersection-Intersection ->}@>   16 |             9656 |           5691
<@\textcolor{blue}{intersection-Intersection ->}@>   17 |            10255 |           6059
<@\textcolor{blue}{intersection-Intersection ->}@>   18 |            10843 |           6428
<@\textcolor{blue}{intersection-Intersection ->}@>   19 |            11448 |           6791
<@\textcolor{blue}{intersection-Intersection ->}@>   20 |            12025 |           7140
<@\textcolor{blue}{intersection-Intersection ->}@>   21 |            12617 |           7499
<@\textcolor{blue}{intersection-Intersection ->}@>   22 |            13225 |           7867
<@\textcolor{blue}{intersection-Intersection ->}@>   23 |            13807 |           8223
<@\textcolor{blue}{intersection-Intersection ->}@>   24 |            14400 |           8580
<@\textcolor{blue}{intersection-Intersection ->}@> 
<@\textcolor{blue}{intersection-Intersection ->}@> -------------------------------------------
<@\textcolor{blue}{intersection-Intersection ->}@> ---------- SIMULATION STATISTICS ----------
<@\textcolor{blue}{intersection-Intersection ->}@> -------------------------------------------
<@\textcolor{blue}{intersection-Intersection ->}@> Traffic Light | Total Cars | Largest Backup
<@\textcolor{blue}{intersection-Intersection ->}@> --------------+------------+---------------
<@\textcolor{blue}{intersection-Intersection ->}@>              0 |       14400 |              18
<@\textcolor{blue}{intersection-Intersection ->}@>              1 |        8581 |               7
<@\textcolor{blue}{intersection-Intersection ->}@> Destroying Intersection...
<@\textcolor{mauve}{car\_generator-Car Generator 1 ->}@> Destroying Car Generator 1...
<@\textcolor{mauve}{car\_generator-Car Generator 0 ->}@> Destroying Car Generator 0...
<@\textcolor{dkgreen}{traffic\_light-Traffic Light 0 ->}@> Destroying Traffic Light 0...
<@\textcolor{dkgreen}{traffic\_light-Traffic Light 1 ->}@> Destroying Traffic Light 1...
Simulation is complete, simulated time: 86.4 Ks
\end{lstlisting}

  \section{Extensibility}

    \subsection{Communication}
    As mentioned in Section \ref{sec:ipc}, it is possible to integrate additional IPC protocols to
    the interface by implementing a derived class of \lstinline{sigutils::SignalIO} with customized
    sending and receiving methods. The base \lstinline{sigutils::SignalIO} class provides methods of
    serializing and deserializing the data structures utilized within the black box interface. The
    derived sending and receiving methods would have to simply implement their specific approaches
    of reading and flushing buffers.

    \subsection{Interface}

    This entire paper focuses on the interoperability established between SST and SystemC processes.
    However, the concept was derived off of the efforts already established by the SST-PyRTL project
    \cite{pyrtl-sst}. In fact, a hybrid version of the Traffic Intersection Simulation has been
    implemented where both SystemC and PyRTL take control over one traffic light using the same IPC
    method. The black box SST component had to be provided distinct instructions to spawn and
    communicate with a SystemC and a non-SystemC process. Meanwhile, the other side of the black box
    interface consisted of a SystemC driver and a PyRTL driver with their respective native
    configurations for establishing communication with the parent process.

    This extensibility was possible due to the generic structure of the black box interface. In
    theory, the interface is able to establish interoperability between SST and almost any other
    synthesizable HDL.


  \begin{thebibliography}{5}

    \bibitem{sst} SST Simulator - The Structural Simulation Toolkit. \texttt{sst-simulator.org}.

    \bibitem{sysc} 1666-2011 - IEEE Standard for Standard SystemC Language Reference Manual. IEEE, 9
    Jan. 2012, \texttt{standards.ieee.org/standard/1666-2011.html}.

    \bibitem{pyrtl-sst} Mrosky, Robert P, et al.
    \textit{Creating Heterogeneous Simulations with SST and PyRTL}.

    \bibitem{msgpack} ``MessagePack.'' \textit{MessagePack: It's like JSON. but Fast and Small.,}
    msgpack.org/index.html.

  \end{thebibliography}

\end{document}

% could you describe the interface in more detail? 
% The diagrams are very useful, but maybe add some more detail about using message pack to compress the data, how you could use the generic structure for various languages besides systemc, etc.
% Possibly also some details of what the boilerplate black box provides.
